\documentclass[12pt,letterpaper]{article}

\usepackage[spanish]{babel}
\usepackage[utf8]{inputenc}
\usepackage{graphicx}
\usepackage{pdfpages}
\usepackage[hidelinks]{hyperref}
\usepackage{amsbsy,amsmath,amstext,amssymb,amsfonts,amsthm}
\usepackage[usenames,dvipsnames]{color}
\usepackage{listings}
\usepackage{verbatim}
\usepackage{coffee4}
\usepackage{tikz}

%You can use \cofe[ABCD]m{alpha=[0,1]}{scale=[0:...]}{algle=[1,360]}{xoff}{yofff}

\newcounter{problem}
\newcommand\Problem{
    \stepcounter{problem}
    \textbf{Problema \theproblem\\[2.0pt]}
}

\newtheorem{theorem}{Teorema}[section]
\newtheorem{corollary}{Corolario}[theorem]
\newtheorem{lemma}[theorem]{Lema}
\newtheorem{definition}[theorem]{Definición}

\renewcommand\qedsymbol{$\blacksquare$}

\title{Tarea 2 \-- Compresión de Datos}
\author{José Joaquín Zubieta Rico\\Lic. Computación \-- UGTO}
\date{\today}

\begin{document}

\maketitle

{\Problem} Tenemos que su código de Hoffman corresponde al generado por el siguiente árbol
\\
\begin{center}
    \begin{tikzpicture}
        \node[circle,draw]{$30$}
        	child[missing]{}
        	child{node[circle,draw]{40} 
        		child{node[circle,draw] {20}} 
        		child{node{}
        			child[missing]{}
        			child{node[circle,draw]{30}}}};
    \end{tikzpicture}
\end{center}


\end{document}
